\section{معماری‌های مختلف}
\begin{frame}{مقدمه}
\begin{itemize}\itemr
\item[-]
اگر شما الان بخواهید یک کامپیوتر بخرید، از بین معماری‌های مختلف دو معماری اصلی پیش روی شما هستند:
\begin{enumerate}\itemr
\item \lr{x86}
\item \lr{ARM}
\end{enumerate}
\end{itemize}
\end{frame}

\begin{frame}{معماری \lr{x86}}
\begin{itemize}\itemr
\item[-]
این معماری بر پایه‌ی معماری 
\lr{Intel 8008}
که در سال ۱۹۷۲ معرفی شد، است.
\item[-]
در واقع کد‌هایی که برای این معماری نوشته شده‌اند را می‌توان برای آخرین 
\lr{CPU}های
\lr{Intel} 
یا
\lr{AMD}
اسمبل و اجرا کرد.
\item[-]
پس از
\lr{Intel 8008}
معماری‌های
\lr{Intel 8088}،
\lr{16-bit 8086} و سپس
\lr{80186}،
\lr{80286} 
و... معرفی شدند و در کل 
\lr{x86}
نام گرفتند.
\item[-]
پردازنده‌های شرکت‌های 
\lr{Intel}
و 
\lr{AMD}
همگی بر پایه این معماری هستند.
\end{itemize}
\end{frame}

\begin{frame}{معماری \lr{ARM}}
\begin{itemize}\itemr
\item[-]
نام این معماری برگرفته از 
\lr{\textbf{A}dvanced \textbf{R}ISC \textbf{M}achines}
که قبل‌تر از
\lr{\textbf{A}corn \textbf{R}ISC \textbf{M}achine}
گرفته شده بود، است.
\item[-]
پردازنده‌های آرم، بخاطر 
\begin{itemize}\itemr
\item[-]
قیمت ارزان،
\item[-]
مصرف انرژی کم و
\item[-]
تولید گرمای کم
\end{itemize}
\begin{flushright}
برای دستگاه‌های سبک و دارای باتری، مثل تلفن‌های هوشمند و لپ‌تاپ‌ها بسیار مناسب هستند.
\end{flushright}

\item[-]
بین سال‌های ۲۰۲۰ تا ۲۰۲۲ سریع‌ترین سوپرکامپیوتر دنیا 
\lr{(Fugako)}
هم از پردازنده‌های معماری آرم استفاده می‌کرد.
\item[-]
چیپ‌های سری \lr{M} شرکت اپل هم از معماری آرم استفاده می‌کنند.
\end{itemize}
\end{frame}