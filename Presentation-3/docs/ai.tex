\section{هوش مصنوعی و معماری کامپیوتر}
\begin{frame}{مقدمه}
\begin{itemize}\itemr
\item[-]
در دهه‌های ۸۰ و ۹۰ میلادی، کامپیوتر‌‌ها هر ۱۸ تا ۲۴ ماه (قانون مور) سریع‌تر می‌شدند.
\item[-]
این یعنی اگر شما امسال یک کامپیوتر می‌خریدید و دوستان شما یک سال بعد از شما کامپیوتر جدیدی می‌خریدند، کامپیوتر‌ آنها بسیار سریع‌تر می‌بود
\item[-]
اما امروزه، تنها راه پیشرفت در معماری کامپیوتر ساخت سخت‌افزار برای یک کاربرد خاص است.
\item[-]
برای مثال پردازند‌ه‌ها گرافیکی
\lr{(GPU)}
برای انجام محاسبات گرافیکی بسیار کارامد هستند. آنها می‌توانند میلیون‌ها ضرب ماتریس در یک هر ثانیه انجام بدهند.
\end{itemize}
\end{frame}

\begin{frame}{پردازنده‌ی \lr{TPU}}
\end{frame}