\section{معماری کامپیوتر - دیروز تا امروز}
\begin{frame}{تکامل معماری کامپیوتر}
\begin{itemize}\itemr
\item[-]
در دنیای امروزی کامپیوتر‌ها برای اهداف زیاد و توسط افراد زیادی استفاده می‌شوند،
\item[-]
کارها و اتفاقاتی که زمانی غیر قابل تصور بود، برای جامعه‌ی ما بسیار بدیهی و مرسوم است،
\item[-]
تکنولوژی معماری‌ کامپیوتر در طول سالیان متمادی، عمدتا به دلیل پیشرفت‌های تکنولوژی ساخت قطعات الکترونیکی، پیشرفت علوم کامپیوتر و نیاز‌‌های افراد پیشرفت کرده است.
\end{itemize}
\end{frame}

\begin{frame}{نسل اول کامپیوتر‌ها}
\begin{itemize}\itemr
\item[-]
در سال ۱۹۳۷، اولین کامپیوتر با استفاده از لامپ‌های خلاء توسط پروفسور ایکن اختراع شد.
\item[-]
در سال ۱۹۴۷، دانشگاه پنسیلوانیا کامپیوتری به نام 
\lr{ENIAC}
را طراحی کرد که از مبنای دودویی برای نمایش اطلاعات استفاده می‌کرد.
\item[-]
معماری کامپیوتر‌های این دوره (و تمام دوره‌ها،) بر اساس مدل 
\lr{Von Neumann}
بود (و هست،) که شامل 
\begin{enumerate}\itemr
\item 
واحد‌ حافظه،
\item 
واحد پردازش، 
\item 
واحد کنترل و
\item 
واحد ورودی/خروجی 
\end{enumerate}
می‌شود.
\end{itemize}
\end{frame}

\begin{frame}{نسل دوم کامپیوتر‌ها}
\begin{itemize}\itemr
\item[-]
در دهه‌ي ۱۹۵۰، ترانزیستور‌ها به جای لامپ‌های خلاء در کامپیوتر‌ها استفاده شدند،
\item[-]
این باعث کاهش حجم و افزایش سرعت کامپیوتر‌ها شد.
\item[-]
در این دوره کامپیوتر‌های دیجیتال و مینی‌کامپیوترها شروع به ظهور کردند
\end{itemize}
\end{frame}

\begin{frame}{نسل سوم کامپیوتر‌ها}
\begin{itemize}\itemr
\item[-]
در دهه‌ی ۱۹۶۰، مدار‌های مجتمع 
\lr{(IC)}
جایگزین ترانزیستور‌ها شدند.
\item[-]
استفاده از \lr{IC}ها باعث افزایش قابلیت پیچیدگی و کارای کامپیوتر‌ها شد.
\item[-]
این به این معنی‌ست که تعداد بیشتری ترانزیستور‌ها را در یک تراشه کوچک‌تر قرار دادند و این امر به کامپیوتر امکان انجام محاسبات پیچیده‌تر و سریع‌تر را می‌داد.
\item[-]
کامپیوتر‌های این دوره (و دوره‌‌های بعدی) از معماری مجموعه دستورات 
\lr{(Instruction Set Architecture)}
استفاده می‌کردند.
\item[-]
معماری کامپیوتر 
\lr{IBM 360}
از معماری‌های مشهور این دوره است.
\end{itemize}
\end{frame}
