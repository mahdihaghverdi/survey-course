\section{Parsing - ``Words" to ``Sentence"}
\begin{frame}{Parsing - ``Words" to ``Sentence"}
\begin{itemize}
{\LARGE \item[-] Take the words and make sentences out of them}
\end{itemize}
\end{frame}

\begin{frame}{Parsing - ``Words" to ``Sentence" (Cont'd)}
\begin{itemize}
\item[-]<1-> Now we have broken everything into words, we can care about how to structure our sentences
and make them meaningful following specific grammer rules. 

\item[-]<2->
In parsing we use a grammer
to define a structure, you can check Python grammer in \url{https://github.com/python/cpython/tree/main/Grammar}
\end{itemize}
\end{frame}

\begin{frame}[fragile]{Parsing - ``Words" to ``Sentence" (Cont'd)}
\begin{flushleft}
This is a piece of Python grammer (version 3.9)
\begin{lstlisting}[numbers=none]
stmt: simple_stmt | compound_stmt
simple_stmt: small_stmt (';' small_stmt)* [';'] NEWLINE
small_stmt: (expr_stmt | del_stmt | pass_stmt | flow_stmt |
import_stmt | global_stmt | nonlocal_stmt | assert_stmt)
3
del_stmt: 'del' exprlist
pass_stmt: 'pass'
flow_stmt: break_stmt | continue_stmt | return_stmt | raise_stmt |
yield_stmt
break_stmt: 'break'
\end{lstlisting}
\end{flushleft}
\end{frame}

\begin{frame}{Parsing - ``Words" to ``Sentence" (Cont'd)}
\begin{itemize}
\item[-]<1> Before python 3.9’s PEG parser, Python parser was a LL(1) parser

\item[-]<2> It was probably was oldest python code which was written by “Guido van Rossum” and hadn’t changed way back since
Decemeber of 1998 :-)
\end{itemize}
\end{frame}