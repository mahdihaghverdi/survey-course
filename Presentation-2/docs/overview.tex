\section{Overview}
\begin{frame}{Overview}
\begin{itemize}
\item[-]
Which steps does CPython take to compile your source code?
\item[-]
Why these steps?
\item[-]
How they are done?
\end{itemize}
\end{frame}

\subsection{Diagram}
\begin{frame}{Diagram}
\begin{itemize}
\item[-] 
{\small \ttfamily 
----------------------------------------------------------- \\
| Decoding -> Tokenizing -> Parsing -> AST | -> Compiling | \\
----------------------------------------------------------- \\
}

\item[-]
Front-end: Decoding, Tokenizing, Parsing and AST

\item[-]
Back-end: Compiling
\end{itemize}
\end{frame}

\subsection{Explantion}
\begin{frame}{Explanation}
\begin{itemize}
\item[-]<1->
We’ve got a front-end and a back-end part in this process.

\item[-]<2->
Front-end: Getting down to the AST

\item[-]<3->
Back-end: Get the generated AST and compile it down to something

\item[-]<4->
Good example is \href{https://www.pypy.org/}{PyPy} which is a front-end for Python

\item[-]<5->
Ease of writing the code

\item[-]<6->
A better view to the process
\end{itemize}
\end{frame}
