\section{مقدمه}

\begin{frame}{لاتک چیست؟}
\begin{itemize}\itemr
\item[-]
نرم‌افزار 
\lr{\LaTeX}
یک سیستم حروف‌چینی‌ مبتنی بر 
\lr{\TeX}
می‌باشد.

\item[-]
پروفسور دونالد کنوث، در سال ۱۹۷۸ برای اولین بار این سیستم حروف‌چینی را معرفی کرد.

\item[-]
پس از انتشار، 
\lr{\TeX}
توسط افراد مختلف تغییراتی روی آن اعمال شد و بسته‌‌های گوناگونی به آن اضافه گردید.

\item[-]
بیشترین تغییرات توسط پروفسور لزلی لمپورت روی 
\lr{\TeX}
اعمال گردید و نام 
\lr{\LaTeX}
بر آن نهاده شد.
\end{itemize}
\end{frame}

\begin{frame}{چرا لاتک؟}
\begin{itemize}\itemr
\item[-]
نرم‌افزار
\lr{\LaTeX}
یک نرم‌افزار رایگان و متن‌باز است و شما می‌توانید براحتی آن‌ها را دانلود، نصب و از آنها استفاده کنید.

\item[-]
کیفیت خروجی 
\lr{\LaTeX}
از اغلب دیگر نرم‌افزار‌های حروف‌چین بیشتر است.

\item[-]
نماد‌ها تخصصی رشته‌های ریاضی، فیزیک، شیمی و علوم مهندسی بسیار گسترده هستند و امکان استفاده از همه‌ی آنها در 
\lr{Micro\$oft Word}
نیست و یا خیلی سخت است اما با لاتک براحتی قابل استفاده‌اند.

\item[-]
مستندات تولید شده با لاتک پایدار هستند. این یعنی اگر فایل 
\lr{source code}
یک سند را از این کامپیوتر به کامپیوتر دیگری منتقل کنیم و آنجا خروجی بگیریم و دو خروجی را مقایسه کنیم، هیچ تغییری در آن ایجاد نمی‌شود.
\end{itemize}
\end{frame}

\begin{frame}{چرا لاتک؟}
\begin{itemize}\itemr
\item[-]
تهیه‌ی فهرست مطالب، نمایه، فهرست واژگان، تصاویر و جداول، ارجاع‌دهی به مراجع و منابع و فهرست مراجع و منابع و... به راحت‌ترین روش ممکن در لاتک امکان‌پذیراند.

\item[-]
بسیاری از مجله‌های تخصصی (بخصوص در رشته‌ی ریاضی) تنها مقالاتی را برای چاپ قبول می‌کنند که با لاتک نوشته شده باشند.
\end{itemize}
\end{frame}

\begin{frame}{ابزار‌های لازم}
\begin{itemize}\itemr
\item[-]
نرم‌افزار 
\lr{TexLive}
که از آدرس
\url{https://tug.org/texlive}
قابل دانلود است.

\item[-]
برای سیستم عامل مک، باید 
\lr{MacTex}
نصب شود.

\item[-]
یک ویرایشگر لاتک؛ می‌توان از دو ویرایشگر 
\lr{TexWorks}
یا 
\lr{TeXstudio}
استفاده کرد که پشتیبانی 
\lr{TeXstudio}
از زبان‌ فارسی بسیار خوب است و از آدرس
\url{http://www.texstudio.org}
قابل دانلود است.

\item[-]
نصب یک فونت استاندارد مانند 
\lr{HM XKayhan}
یا 
\lr{Yas}

\end{itemize}
\end{frame}