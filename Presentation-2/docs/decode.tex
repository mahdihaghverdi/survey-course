\section{Decoding - ``Bytes" to ``Text"}
\begin{frame}{Decoding - ``Bytes" to ``Text"}
\begin{itemize}
{\LARGE \item[-] Translate bytes from disk to actual text}
\end{itemize}
\end{frame}

\subsection{Encoding Declaration}
\begin{frame}{Encoding Declaration}
\begin{itemize}
\item[-] As of \href{https://peps.python.org/pep-0263/}{PEP 263}, you can specify the encoding of your Python module (basically a module is a text
file which python code is written into) at the very top line of the file something like:
\end{itemize}
\end{frame}

\begin{frame}[fragile]{Encoding Declaration (Cont'd)}

\begin{flushleft}
Declaration:
\begin{lstlisting}
#!/usr/bin/python
# -*- coding: <encoding name> -*-
\end{lstlisting}
\end{flushleft}

\begin{flushleft}
e.g.
\begin{lstlisting}[language=python, keywordstyle=\color{Mulberry}\textbf]
#!/usr/bin/python
# -*- coding: ascii -*-

import math
print(math.sin(math.radians(90)))  # 1.0
\end{lstlisting}
\end{flushleft}
\end{frame}

\begin{frame}[fragile]{Encoding Declaration (Cont'd)}
\begin{flushleft}
Which gets compiled like this:
\begin{lstlisting}[language=python]
re.compile("conding[:=]\s*([-\w.]+)")
\end{lstlisting}
\end{flushleft}
\end{frame}

\subsection{Default encoding and Non-ASCII characters}
\begin{frame}{Default Encoding and Non-ASCII Characters}
\begin{itemize}
\item[-]<1>
From \href{https://peps.python.org/pep-3120/}{PEP 3120} UTF-8 is considered as the default enconding, and along with this with \href{https://peps.python.org/pep-3131/}{PEP 3131}

\item[-]<2>
Python supports Non-ASCII identifiers also, this means that you can use french or germen alphabet
(with accent) in your variable names, like:
\end{itemize}
\end{frame}

\begin{frame}[fragile]{Default Encoding and Non-ASCII Characters (Cont'd)}
\begin{lstlisting}[language=python, keywordstyle=\color{Mulberry}\textbf]
löwis = 'Löwis'
print(löwis)
\end{lstlisting}
\end{frame}
