\section{ساختار یک کامپایلر}
\begin{frame}{ساختار کلی}
\begin{itemize}\itemr
\item[-]
تا کنون به کامپایلر به عنوان یک جعبه دارای ورودی خروجی نگاه می‌کردیم،
\item[-]
اما اگر این جعبه را باز کنیم، با دو قسمت اصلی در کامپایلر‌ها مواجه می‌شویم:
\begin{enumerate}\itemr
\item 
آنالیز
\item 
سنتز
\end{enumerate}
\end{itemize}
\end{frame}

\begin{frame}{آنالیز}
\begin{itemize}\itemr
\item[-]
این قسمت، \lr{source code} را به قسمت‌های مختلفی می‌شکند،

\item[-]
و قواعد گرامی را به قسمت‌های مختلف تحمیل می‌کند.

\item[-]
این قسمت، اگر قسمتی را مخالف قوانین و گرامر زبان پیدا کند، پیام‌های مطلع کننده‌ای به کار نشان می‌دهد که ورودی را اصلاح کند.

\item[-]
و در پایان این قسمت، داده ساختاری به نام \lr{\textit{symbol table}} را تولید می‌کند که تقریبا در تمامی گام‌های کامپایل (که در ادامه به آنها پرداخته می‌شود،) استفاده می‌شود.
\end{itemize}
\end{frame}

\begin{frame}{سنتز}
\begin{itemize}\itemr
\item[-]
قسمت سنتز، بعد از رد شدن از تمامی مراحل قسمت آنالیز، خروجی مورد نیاز ما را تولید می‌کند.

\item[-]
به قسمت آنالیز 
\lr{front-end}
و قسمت سنتز
\lr{back-end}
هم گفته می‌شود.
\end{itemize}
\end{frame}

\begin{frame}{\lr{front-end}}
\begin{figure}[H]
\begin{center}
\includegraphics[width=0.5\textwidth, height=0.8\textheight, angle=-0.5]{docs/images/front}
\end{center}
\end{figure}
\end{frame}

\begin{frame}{\lr{back-end}}
\begin{figure}[H]
\begin{center}
\includegraphics[width=0.5\textwidth, height=0.6\textheight, angle=-0.5]{docs/images/back}
\end{center}
\end{figure}
\end{frame}
