%   Borrowed from A. Shafiei
%
%   Copyright (c)  2023  A. Shafiei
%   Permission is granted to copy, distribute and/or modify this document
%    under the terms of the GNU Free Documentation License, Version 1.3
%    or any later version published by the Free Software Foundation;
%    with no Invariant Sections, no Front-Cover Texts, and no Back-Cover Texts.
%    A copy of the license is included in the section entitled "GNU
%    Free Documentation License".

\documentclass[aspectratio=169, dvipsnames, svgnames, x11names]{beamer}

% URLs and hyperlinks ---------------------------------------
\usepackage{hyperref}
\hypersetup{
    colorlinks=true,
    linkcolor=NavyBlue,
    filecolor=magenta,      
    urlcolor=blue,
}
\usepackage{xurl}
%---------------------------------------------------

\usefonttheme{serif}
\usepackage{graphicx}
\usepackage{amsfonts}
\usepackage{mathtools, nccmath}
\usepackage{amssymb, amsmath}
\usepackage{xspace}
\usepackage{tikz}
\usepackage{standalone}
\usepackage{euler}
\usepackage{color,xcolor}
\usepackage{fontspec}
\usepackage{nameref}
\usepackage{manfnt}
\usepackage{listings}
\usepackage{xcolor}
\usepackage{algorithm}
\usepackage[noend]{algpseudocode}
\usepackage{algorithmicx}
\usepackage{docs/style}

\input{docs/macros}
\usetikzlibrary{arrows,calc}
\usetikzlibrary{positioning,shapes,chains,fit}

\tikzset{
    %Define style for boxes
    node/.style={
        circle,
        draw=black, thick,
        align=center,
    },
    ss/.style={
        circle,
        draw=black,
        align=center,
    },
    io/.style={
        rounded corners,
        draw=black,
        align=center,
    },
    proc/.style={
        rectangle,
        draw=black,
        align=center,
    },
    ifelse/.style={
	ellipse,
	draw=black,
	align=center,
    },
    cond/.style={
	diamond,
	aspect = 3,
	draw=black,
	align=center,
    },
    cloudy/.style={
	cloud,
	cloud puffs=12,
	cloud ignores aspect,
	align=center,
	draw=black,
    },
    txt/.style={
        draw = none,
        align = center,
        font = \footnotesize,
    },
    coin/.style={
        rectangle,
        minimum height=1mm,
        minimum width=1cm,
        draw=black,
        fill=black!20,
        rounded corners
    },
    towercolor/.style={
        fill=black!80
    },
    towerbase/.style={
        trapezium,
        trapezium angle=75,
        trapezium stretches=true,
        towercolor,
        minimum width=7mm,
        minimum height=2mm,
    },
    tower/.style={
        rectangle,
        rounded corners,
        towercolor,
        minimum width=2mm,
        minimum height=26mm,
    },
}


\title{How CPython Compiler Works}
\author{Mahdi Haghverdi}

\institute{
\\\includegraphics[height=1.2cm]{logos/logo}\\
Isfahan University
}
\date{}

\begin{document}
    
\begin{frame}[plain]
\begin{center}
In the name of Allah
\end{center}
\maketitle
\end{frame}

\setcounter{framenumber}{0}
\raggedleft

\begin{frame}{Content}
\begin{flushleft}
\tableofcontents
\end{flushleft}
\end{frame}

\section{Overview}
\begin{frame}{Overview}
\begin{itemize}
\item[-]
Which steps does CPython takes to compile your source code?
\item[-]
Why these steps?
\item[-]
How they are done?
\end{itemize}
\end{frame}

\subsection{Diagram}
\begin{frame}{Diagram}
\begin{itemize}
\item[-] 
{\small \ttfamily 
----------------------------------------------------------- \\
| Decoding -> Tokenizing -> Parsing -> AST | -> Compiling | \\
----------------------------------------------------------- \\
}

\item[-]
Front-end: Decoding, Tokenizing, Parsing and AST

\item[-]
Back-end: Compiling
\end{itemize}
\end{frame}

\subsection{Explantion}
\begin{frame}{Explanation}
\begin{itemize}
\item[-]
We’ve got a front-end and a back-end part in this process.

\item[-]
Front-end: getting down to the AST

\item[-]
Back-end: to get the generated AST and compile it down to something

\item[-]
Good example is \href{https://www.pypy.org/}{PyPy} which is a front-end for Python

\item[-]
Ease of writing the code

\item[-]
A better view to the process
\end{itemize}
\end{frame}

\section{Decoding - ``Bytes" to ``Text"}
\section{Tokenizing}
\section{Parsing}
\section{Abstract Systax Tree}
\section{Compiling}
\end{document}
