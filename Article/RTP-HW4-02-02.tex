\documentclass[11pt, twoside]{imsproc}

\usepackage[inline]{enumitem}

\usepackage{graphicx}
\usepackage{geometry}
\usepackage{subfigure}
\usepackage{booktabs}
\usepackage{float}
\usepackage[colorlinks,citecolor=blue]{hyperref}
\setcounter{page}{1}
\geometry{left=2.5cm,right=2.5cm,top=3cm,bottom=3cm,headsep=1.1cm}
\footskip 0.7cm

%------------------------------------------------------------------------------------%
\newcommand*{\publname}{
\begin{tabular}{c}
\includegraphics[width=1.8cm]{UI-Logo.jpg}\\
\url{http://www.ui.ac.ir}
\vspace{0.2cm}
\end{tabular}
\hfill
\begin{tabular}{c}\toprule
\vspace{0.1cm}
\scriptsize \bfseries روش پژوهش و ارائه\\
\vspace{0.1cm}
\copyright{} \lr{1xxx} دانشگاه اصفهان \\ \bottomrule
\end{tabular}
\hfill
\begin{tabular}{c}
\includegraphics[width=2.2cm]{JMS-Logo.jpg}\\
\url{http://math-sci.ui.ac.ir}
\vspace{-0.2cm}
\end{tabular}
}
%------------------------------------------------------------------------------------%

\usepackage[para*]{manyfoot}
\SetFootnoteHook{\setLTR}
\DeclareNewFootnote[para]{A}
\usepackage{xepersian}
\makeatletter
\let\c@footnoteA\c@footnote
\makeatother
\let\LTRfootnote\footnoteA
\AtBeginDocument{\label{firstpage}}
\AtEndDocument{\label{lastpage}}
\settextfont[Scale=1.1]{XB Niloofar.ttf}
\setlatintextfont [Scale=0.9]{times new roman.ttf}
\linespread{1.35}
\newsavebox\uilogo
\newsavebox\nologo
\sbox\uilogo{\includegraphics[width=0.8cm]{UI-Logo.pdf}}
\sbox\nologo{\includegraphics[width=1cm]{JMS-Logo.jpg}}
\makeatletter
\def\seriesno#1{\gdef\@seriesno{#1}}
\def\issueno#1{\gdef\@issueno{#1}}
\def\publicationname#1{\gdef\@publicationname{#1}}
\def\ps@ijheadings{\ps@empty
  \def\@evenhead{%
   \parbox{\textwidth}{%
    \setTrue{runhead}%
    \normalfont\scriptsize
    \usebox\uilogo\hfill
    \def\thanks{\protect\thanks@warning}%
  \leftmark{}{}, \@publicationname/
    جلد x،
   % \@seriesno{}
    شماره x 
   % \@issueno
 (1xxx) \pageref{firstpage}--\pageref{lastpage}
    \hfill
   \usebox\nologo\vskip0pt
     \vskip-7pt
     \rule{\textwidth}{0.5pt}
      \vskip-12pt
        \rule{\textwidth}{0.5pt}
    }}
  \def\@oddhead{%
   \parbox{\textwidth}{%
    \setTrue{runhead}%
    \normalfont\scriptsize
    \usebox\uilogo\hfill
    \def\thanks{\protect\thanks@warning}%
    \rightmark{}{}, \@publicationname/
    جلد x،
    %\@seriesno{}
    شماره x 
    %\@issueno
 (1xxx) \pageref{firstpage}--\pageref{lastpage}
    \hfill
   \usebox\nologo\vskip0pt
     \vskip-6pt
     \rule{\textwidth}{0.5pt}
      \vskip-12pt
        \rule{\textwidth}{0.5pt}
    }}
   \def\@evenfoot{\normalfont\small\thepage
     \hfill \scriptsize{}\hfill}
    \def\@oddfoot{\normalfont\small\hfill\scriptsize{}\hfill\thepage}
 }%   
\def\enddoc@text{%
\ifx\@empty\@translators \else\@settranslators\fi
  \ifx\@empty\addresses \else\@setaddresses\fi}
  
\renewcommand*{\@makefnmark}{\hbox{\@textsuperscript{\normalfont\@thefnmark}}}
  \def\MFL@fnotepara#1#2#3{\let\@thefnmark\@empty
    \NCC@makefnmark{\latinfont #2}%
    \MFL@insert#1{\reset@font\footnotesize
      \ifx\@thefnmark\@empty \@tempswafalse \else
        \@tempswatrue
        \protected@edef\@currentlabel{\@thefnmark}%
      \fi
      \color@begingroup
        \if@tempswa
          \setbox\@tempboxa\hbox{\@makefnmark}%
          \ifMFL@paraindent
            \@tempdima.8em \advance\@tempdima-\wd\@tempboxa
            \ifdim \@tempdima<\z@ \@tempdima\z@ \fi
          \else
            \@tempdima\z@
          \fi
        \fi
        \setbox\@tempboxa\hbox{%
          \if@tempswa
            \hskip\@tempdima\unhbox\@tempboxa\nobreak
          \fi
          \ignorespaces\resetlatinfont#3\unskip\strut
          \ifMFL@split \penalty\m@ne\space \else
            \penalty-10 \hskip\footglue
          \fi
        }%
        \dp\@tempboxa\z@ \ht\@tempboxa\MFL@fudgefactor\wd\@tempboxa
        \box\@tempboxa
      \color@endgroup
    }%
  }
\long\def\@footnotetext#1{\insert\footins{%
   \if@RTL@footnote\@RTLtrue\else\@RTLfalse\fi%
    \reset@font\tiny
    \interlinepenalty\interfootnotelinepenalty
    \splittopskip\footnotesep
    \splitmaxdepth \dp\strutbox \floatingpenalty \@MM
    \hsize\columnwidth \@parboxrestore
    \protected@edef\@currentlabel{%
       \csname p@footnote\endcsname\@thefnmark
    }%
    \color@begingroup
      \@makefntext{%
        \rule\z@\footnotesep\ignorespaces\if@RTL@footnote#1\else\latinfont#1\fi\@finalstrut\strutbox}%
    \color@endgroup}}%
%\footdir@temp\footdir@my@ORG@xepersian@footnotetext\@footnotetext{\bidi@footdir@footnote}%
\makeatother
\pagestyle{ijheadings}
\seriesno{1}
\issueno{1}
\publicationname{}
\title{معماری کامپیوتر \lr{(Computer Architecture)}}
\author{
فاطمه علی‌ملکی، امیررضا جهانگیری، محمدحسین چهکندی، مهدی حق‌وردی و خدیجه نظری
}

\thanks{
عبارات و کلمات کليدي: {معماری کامپیوتر، واحد پردازش، \lr{x86}، \lr{arm}، \lr{qbit}}\\
دبیرتخصصی رابط: استاد دکتر فریا نصیری مفخم\\  
نوع مقاله: تکلیف کلاسی\\
تاریخ دریافت: 1400/9/14
\quad
تاریخ پذیرش: 1400/10/18 
}
\copyrightinfo{}{(دانشگاه اصفهان)}

\makeatletter
\def\ps@firstpage{\ps@plain
\def\@oddfoot{\normalfont\small\hfil\thepage\hfil
\global\topskip\normaltopskip}%
\let\@evenfoot\@oddfoot
\def\@oddhead{\@serieslogo\hss}%
\let\@evenhead\@oddhead % in case an article starts on a left-hand page
}
%%%%%%%%%%%%%%%%%
\settextfont[Scale=1.1]{Yas}
%%%%%%%%%%%%%%%%%%%%%%%
\makeatother
\begin{document}
\begin{abstract}
در این مقاله، به بررسی معماری کامپیوتر، روند توسعه‌ی آن، انواع معماری کامپیوتر و پیشرفت‌های آن می‌پردازیم.
\end{abstract}
\maketitle
\section{مقدمه}

\section{معماری کامپیوتر - دیروز تا امروز}
\vskip 0.4 true cm

در دنیایی که ما امروزه می‌شناسیم رایانه‌ها برای اهداف زیاد و توسط افراد زیادی استفاده می‌شوند. آنها قادرند چیزهای جذابی را به نمایش بگذارند که ذهن مارا متحیر میکند. اتفاقی که زمانی غیر قابل تصور بود برای جامعه ما مرسوم است.

تکنولوژی معماری کامپیوتر در طول سالیان متمادی تکامل یافته است. تغییرات در معماری کامپیوتری عمدتاً به دلیل پیشرفت‌های تکنولوژی ساخت قطعات الکترونیکی، نیاز‌های کاربران و پیشرفت علوم رخ داده است. در ادامه به خلاصه‌ای از تکامل معماری کامپیوتر از زمان ظهور اولین کامپیوترها تا به امروز می‌پردازیم. 

\begin{enumerate}
\item نسل اول کامپیوترها

در سال ۱۹۳۷، اولین کامپیوتر با استفاده از لامپ‌های خلاء توسط پروفسور ایکن اختراع شد. در سال ۱۹۴۷، دانشگاه پنسیلوانیا کامپیوتری به نام \lr{ENIAC} را طراحی کرد که از مبنای دودویی برای نمایش اطلاعات استفاده می‌کرد.\footnote{مشهورترین نمونه از این دوره می‌باشد} در این دوره، کامپیوترها از لامپ‌های خلاء و رله‌ها برای اجرای عملیات استفاده می‌کردند.
معماری کامپیوترهای این دوره معمولاً به صورت برداری \lr{(Von Neumann)} بود که شامل واحدهای حافظه، واحد پردازش، واحد کنترل و واحد ورودی/خروجی می‌شد.

\item نسل دوم کامپیوترها

در دهه ۱۹۵۰، ترانزیستورها به جای لامپ‌های خلاء در کامپیوترها استفاده شدند. این باعث کاهش حجم و افزایش سرعت کامپیوترها شد. در این دوره، کامپیوترهای دیجیتال شروع به ظهور کردند و از معماری فرمال برای طراحی استفاده می‌شد. این دوره شاهد ظهور کامپیوترهای دیجیتال و کامپیوترهای مینی‌کامپیوتر بود.

\item نسل سوم کامپیوترها

در دهه ۱۹۶۰، مدارهای مجتمع\LTRfootnote{\lr{IC}} جایگزین ترانزیستورها شدند. استفاده از مدارهای مجتمع باعث افزایش قابلیت پیچیدگی و کارایی کامپیوترها شد. به این معنی که تعداد بیشتری ترانزیستور و قطعه الکترونیکی را در یک تراشه کوچک‌تر قرار دادند. این امر به کامپیوترها امکان انجام عملیات پیچیده‌تر و سریع‌تر را می‌داد. به طور خلاصه، مدارهای مجتمع به کامپیوترها امکانات بیشتری را بخشیدند و آن‌ها را کارآمدتر می‌ساختند. در این دوره، معماری مینی‌کامپیوترها و سوپرکامپیوترها توسعه یافت. در این دوره، مدارهای مجتمع بزرگتر و پیچیده‌تری استفاده شدند. کامپیوترهای این دوره از معماری مجموعه دستورات\LTRfootnote{\lr{Instruction Set Architecture}} استفاده می‌کردند. معماری کامپیوتر \lr{IBM/360} از معماری‌های مشهور این دوره است.

\item نسل چهارم کامپیوترها

در دهه ۱۹۷۰، ریزپردازنده‌ها\LTRfootnote{\lr{Microcontroller}} به جای مدارهای مجتمع استفاده شدند. این باعث افزایش قابلیت انعطاف‌پذیری و کاهش هزینه ساخت کامپیوترها شد. در این دوره، معماری کامپیوترهای شخصی و کامپیوترهای قابل حمل توسعه یافت 

\item نسل پنجم کامپیوترها

در دوره نسل پنجم کامپیوترها، که در دهه 1980 آغاز شد، تحولات مهمی در معماری کامپیوتر رخ داد. در این دوره، دو نوع کامپیوتر مهم به وجود آمدند: کامپیوترهای موازی و کامپیوترهای برداری. کامپیوترهای موازی قدرت پردازش را با استفاده از چندین واحد پردازشگر به صورت همزمان افزایش می‌دادند. این کامپیوترها قابلیت انجام همزمان و همروند بسیاری از عملیات‌ها را داشتند و برای برنامه‌هایی که نیاز به پردازش موازی داشتند، بسیار مناسب بودند.
کامپیوترهای برداری قدرت پردازش را با استفاده از پردازشگرهای برداری بهبود می‌بخشیدند. این پردازشگرها برای عملیات‌های مربوط به بردارها و ماتریس‌ها بهینه شده بودند و برای برنامه‌های علمی و مهندسی که با داده‌های برداری سر و کار داشتند، مناسب بودند.
در این دوره، استفاده از مدارهای مجتمع فوق بزرگ\LTRfootnote{\lr{VLSI}} و مدارهای مجتمع فوق فوق بزرگ\LTRfootnote{\lr{ULSI}} نیز رایج شد. این تکنولوژی‌ها به طراحی و ساخت مدارهای بسیار پیچیده و کوچکتر از نسل‌های قبلی کمک کردند و باعث افزایش کارایی و قدرت پردازش کامپیوترها شدند.
در این دوره، شاهد ظهور کامپیوترهای جدیدی نیز بودیم که از جمله آن‌ها می‌توان به کامپیوترهای شخصی\LTRfootnote{\lr{Personal Computers}} و سیستم‌های توزیع شده\LTRfootnote{\lr{Distributed Systems}} اشاره کرد. کامپیوترهای شخصی با عملکرد مناسب و هزینه کم، به کاربران خانگی و کسب و کارها وارد شدند و سیستم‌های توزیع شده با استفاده از شبکه‌های کامپیوتری، امکان ارتباط و همکاری بین کامپیوترها را فراهم می‌کردند.   

\end{enumerate}

\section{اجزا}
\vskip 0.4 true cm

\begin{itemize}
\item اجزای \lr{CPU}

اگر بخواهیم \lr{CPU} را درنگاه سطح بالا بررسی کنیم، \lr{CPU} از 2 قسمت تشکیل شده:
\begin{enumerate*}
\item \lr{Datapath}
\item \lr{Control Unit}
\end{enumerate*}

همانطور که در درس معماری کامپیوتر به آن اشاره شد، \lr{Datapath} در واقع مسیری‌ست که برای انجام یک دستورالعمل طی می‌شود و این مسیر شامل قسمت‌های مختلفی، از جمله
\lr{Register File}،
\lr{ALU}
برای انجام محاسبات، چندین مالتی‌پلکسر، واحدهای جمع، عملیات \lr{Extend} و... می‌شود.

قسمت \lr{Control Unit}، یا همان واحد کنترل قسمتی است که \lr{Datapath} را کنترل می‌کند و تعیین می‌کند که \lr{Datapath} دستگاه‌های \lr{I/O} و \lr{Memory} چه کاری انجام بدهند.

از نگاه سطح بالاتر، \lr{CPU} شامل ۴ جز اصلی است:
\lr{Memory}،
\lr{Control Unit}،
\lr{Datapath}
و دستگاه‌های ورودی/خروجی.

\begin{enumerate}
\item \lr{Data path}

در این تصویر
\ref{fig:datapath-red}
قسمت‌هایی که در بخش \lr{Datapath} طی می‌شوند با خطوط قرمز رنگ مشخص شده‌اند. همانطور که واضح است، \lr{Datapath} شامل جمع کننده، مالتی پلکسر، رجیستر فایل، \lr{Extend} و \lr{ALU} می‌شود.

\begin{figure}[H]
\begin{center}
\includegraphics*[width=0.9\textwidth, height=0.55\textheight]{images/datapath-red}
\end{center}
\caption{\small{\lr{Datapath} در یک نگاه}}
\label{fig:datapath-red}
\end{figure}

\item \lr{Control Unit}

وقتی برای یک معماری خاص \lr{Instruction Set} نوشته می‌شود، در واقع به این معنی است که هر \lr{Instruction} از یک سری رجیسترهای خاص استفاده کند و همچنین از مسیر خاصی در  \lr{Datapath} رد بشود. این مسیریابی و کد گشایی و کنترل مسیر گذر یک دستور و داده‌های آن، توسط \lr{Control Unit} انجام می‌شود.
تمام کارهایی که کنترل یونیت انجام می‌دهد با رنگ آبی در تصویر
\ref{fig:control-unit-signals}
 مشخص شده‌اند که شامل سیگنال‌هایی است که با فرستادن صفر و 1 تعیین می‌کنند که یک عملی انجام بشود یا نشود و بدین صورت کنترل یونیت \lr{Datapath} را کنترل می‌کند.

\begin{figure}[H]
\begin{center}
\includegraphics*[width=0.9\textwidth, height=0.55\textheight]{images/cu}
\end{center}
\caption{\small{سیگنال‌های \lr{Control Unit} در \lr{Datapath}}}
\label{fig:control-unit-signals}
\end{figure}

\end{enumerate}

\item پردازنده‌ی \lr{AMD Barcelona}

تصویر 
\ref{fig:barcelona-real}
معماری پردازنده چند هسته‌ای به نام \lr{AMD Barcelona microprocessor} می‌باشد که شامل ۴ هسته است. در تصویر
\ref{fig:barcelona-sketch}
هسته اول آن بطور کامل شرح داده شده است که شامل قسمت‌هایی همچون
\lr{Fetch/Decode/EXE}،
\lr{Load/Store}،
بخش حافظه‌ی \lr{cache} و... است که هرچهار هسته‌ی این پردازنده دقیقاً شامل همین قسمت‌هایی هستند که در هسته اول به آن اشاره شد.
\begin{figure}[H]
\begin{center}
\subfigure
[عکس واقعی پردازنده]
{\label{fig:barcelona-real}\includegraphics[width=0.5\textheight, height=0.6\textwidth]{./images/amd-1}
}
\hspace{3mm}
\subfigure
[نقشه‌ی طراحی پردازنده]
{\label{fig:barcelona-sketch}\includegraphics[width=0.5\textheight, height=0.6\textwidth]{./images/amd-2}
}
\end{center}
\label{fig:amd-barcelona}
\end{figure}

\end{itemize}

\section{معماری‌‌های مختلف}
\vskip 0.4 true cm

دو معماری بسیار معروف در بازار کامپیوتر، معماری‌های 
\lr{x86}
و 
\lr{arm}
هستند که در ادامه به بررسی این دو معماری می‌پردازیم.

\begin{itemize}
\item معماری \lr{x86}

در سال ۱۹۷۲ شرکت 
\lr{Intel}
معماری 
\lr{Intel 8008}
را معرفی کرد. در تصویر
\ref{fig:8008}
یکی از پردازنده‌های اولیه ساخته بر پایه این معماری را مشاهده می‌کنید.

\begin{figure}[b]
\begin{center}
\includegraphics[width=0.5\textwidth, height=0.3\textheight]{images/8008}
\end{center}
\caption{یک نوع پردازنده \lr{Intel C8008-1} با سرامیکی بنفش، درب و پین‌های فلزی با روکش طلا.}
\label{fig:8008}
\end{figure}
پس از آن، شرکت اینتل معماری‌های دیگری به نام‌های 
\lr{Intel 8088}،
\lr{Intel 16-bit 8086}،
\lr{Intel 80186}،
\lr{Intel 80286}
و... معرفی کرد و این سری معماری 
\lr{x86}
نام گرفتند. در واقع تمامی پردازنده‌های کنونی همگی بر پایه‌ی این معماری نوشته شده‌اند و کد‌‌هایی که برای آن پردازنده‌های در آن سال‌ها نوشته شده‌اند را می‌توان کامپایل و روی جدیدترین پردازنده‌های کنونی اجرا کرد.

پردازنده‌های شرکت‌های 
\lr{Intel}
و 
\lr{AMD}
همگی برپایه‌ی این معماری ساخته می‌شوند.
\item معماری \lr{arm}

نام این معماری برگرفته از 
\lr{\textbf{A}dvanced \textbf{R}ISC \textbf{M}achines}
که قبل‌تر از 
\lr{\textbf{A}corn \textbf{R}ISC \textbf{M}achines}
گرفته شده‌ بود، است.

پردازنده‌های آرم، به دلیل
\begin{itemize}
\item 
قیمت ارزان،
\item 
مصرف انرژی کم و
\item 
تولید گرمای کم
\end{itemize}
برای دستگاه‌های سبک و دارای باتری مثل تلفن‌‌های هوشمند و لپ‌تاپ‌ها بسیار مناسب هستند.

البته پردازنده‌های آرم پردازنده‌های ضعیفی نیستند و بین سال‌‌های ۲۰۲۰ تا ۲۰۲۲ سریع‌ترین سوپرکامپیوتر دنیا 
\lr{(Fugako)}
هم از پردازنده‌های معماری آرم استفاده می‌کرد.

شرکت 
\lr{Apple}
هم چیپ‌های سری \lr{M} را برپایه معماری آرم طراحی و تولید می‌کند.
\end{itemize}

\section{معماری کامپیوتر در آینده}
\vskip 0.4 true cm
معماری کامپیوتر در آینده بر اساس نیاز‌های فعلی ما ایجاد خواهد شد.

برخی از نیاز‌های اساسی ما در حوزه‌های زیر می‌باشد:
\begin{itemize}
\item هوش مصنوعی و یادگیری ماشین

به عنوان مثال می‌توان به نیاز حسگر‌های قدرتمندتری که دیتا‌های متنوع‌تر و با کیفیت‌تری را دریافت و پردازش کنند اشاره کرد

\item پردازش داده‌های کلان

مغز ما حجم بسیاری از داده‌ها را پردازش می‌کند که نیازمند ساختار پیشرفته‌ای برای مدیریت و پردازش این حجم از داده است. ساختار نورون‌های مغز به شیوه‌ایست که در عین انتقال اطلاعات و فعالیت به عنوان انتقال دهنده‌ی اطلاعات، مسئول ذخیره‌سازی و پردازش اطلاعات نیز هست. الگو برداری از این مدل می‌تواند منجر به ایجاد معماری‌های جدیدی در آینده‌ی نه چندان دور شود.

\item محاسبات کوانتومی

با پیشرفت فناوری کوانتومی، کامپیوترهای کوانتومی قادر به انجام محاسبات با سرعت و کارای بسیار بالا در پردازش داده‌های پیچیده خواهند بود.

\item محاسبات نوروفورمیک

منظور از محاسبات نوروفورمیک، محاسباتی مشابه ساختار مغز انسان است. معماری‌ای که در پردازش موازی قدرتمند بوده و انعطاف‌پذیر است و در عین حال در مصرف انرژی بهینگی بالایی دارد.
\end{itemize}

\textbf{معماری کوانتومی}

این معماری از ساختار‌های کلاسیک بیتی پیروی نکرده و به جای آن، از مفهومی تحت عنوان کیوبیت پیروی می‌کند.

این مفهوم دارای پیچیدگی‌های بیشتری نسبت به ساختار بیتی است. به صورت ساده‌ می‌توان کیوبیت را بیتی در نظر گرفت که در آنِ واحد هم آبی رنگ است و هم قرمز رنگ!

در نتیجه با این معماری می‌توان به دستاورد‌های زیر اشاره کرد:
\begin{enumerate}
\item
سرعت بالا در حل مسائل پیچیده
\item 
توانایی محاسبات مختلف به صورت موازی و همزمان
\item 
دسترسی به رمزنگاری و امنیت بالاتر
\item 
فناوری‌های نوروفورمیک
\end{enumerate}
\section{هوش مصنوعی و معماری کامپیوتر}
\vskip 0.4 true cm

در دهه‌های ۸۰ و ۹۰ میلادی، کامپیوترها هر ۱۸ ماه دو برابر سریع‌تر می‌شدند. این بدان معنا بود که اگر شما یک کامپیوتر می‌خریدید و دوستتان یک سال بعد از شما یک کامپیوتر جدیدتر می‌خرید، کامپیوتر دوستتان بسیار سریع‌تر بود. مهندسان و مردم به این سرعت پیشرفت کامپیوترها عادت کرده بودند.

اما امروزه، در معماری کامپیوتر، می‌دانیم که تنها راه پیشرفت، افزودن شتاب‌دهنده‌هایی است که فقط برای یک کاربرد خاص خوب کار می‌کنند. برای مثال، پردازنده‌های گرافیکی\LTRfootnote{\lr{GPU}} برای انجام محاسبات گرافیکی بسیار کارآمد هستند. آنها می‌توانند میلیون‌ها ضرب ماتریس را در هر ثانیه انجام دهند، که برای کارهایی مانند رندرینگ \lr{3D} و بازی‌های ویدئویی ضروری است.

با کند شدن قانون مور، دیگر انتظار نمی‌رود که کامپیوترهای همه منظوره سریع‌تر شوند. این به این دلیل است که قانون مور، که بیان می‌کند تعداد ترانزیستورها در یک میکروچیپ هر دو سال دو برابر می‌شود، در حال کند شدن است. این به این دلیل است که ساخت ترانزیستورهای کوچکتر و کوچکتر چالش برانگیزتر می‌شود.

با این حال، مردم همچنان به دنبال راه‌هایی برای بهبود عملکرد کامپیوترهای خود هستند. یکی از راه‌ها افزودن شتاب‌دهنده‌هایی است که فقط برای یک کاربرد خاص خوب کار می‌کنند.

به طور اتفاقی، همزمان با این وقایع، یک انقلاب در هوش مصنوعی به نام یادگیری ماشین رخ داده است. یادگیری ماشین بر ضرب ماتریس متکی است. برای مثال، شبکه‌های عصبی مصنوعی \lr{(ANN)} از ضرب ماتریس برای یادگیری الگوها در داده‌ها استفاده می‌کنند.

واحد پردازش تنسور \lr{(TPU)} یک نوع شتاب‌دهنده یادگیری ماشین است که توسط گوگل طراحی شده است. \lr{TPU}ها برای انجام ضرب ماتریس بسیار کارآمد هستند، که برای کارهایی مانند آموزش شبکه‌های عصبی مصنوعی \lr{(ANN)} ضروری است.

\begin{thebibliography}{99}%
\begin{LTRbibitems}
\resetlatinfont

\bibitem{Aho}
A. V. Aho. \emph{Compilers: Principles, Techniques, \& Tools}. Pearson/Addison Wesley, 2007.
\end{LTRbibitems}
\end{thebibliography}

%------------------------------------------------------------------------------------------------
%نوشتن بیوگرافی نویسندگان الزامی می باشد.

\bigskip
\bigskip 

\noindent \rule{.4\linewidth}{0.8pt}\\
\noindent\footnotesize{\bfseries خدیجه نظری} \\
\begin{tabular}{p{14 cm}}
\footnotesize
ورودی ۱۴۰۰ مهندسی کامپیوتر دانشگاه اصفهان
\end{tabular}\\\\
\noindent\footnotesize{\bfseries فاطمه علی‌ملکی} \\
\begin{tabular}{p{14 cm}}
\footnotesize
ورودی ۱۴۰۰ مهندسی کامپیوتر دانشگاه اصفهان
\end{tabular}\\\\
\noindent\footnotesize{\bfseries مهدی‌ حق‌وردی} \\
\begin{tabular}{p{14 cm}}
\footnotesize
ورودی ۱۴۰۰ مهندسی کامپیوتر دانشگاه اصفهان
\end{tabular}\\\\
\noindent\footnotesize{\bfseries امیررضا جهانگیری} \\
\begin{tabular}{p{14 cm}}
\footnotesize
ورودی ۱۴۰۰ مهندسی کامپیوتر دانشگاه اصفهان
\end{tabular}\\\\
\noindent\footnotesize{\bfseries محمدحسین چهکندی} \\
\begin{tabular}{p{14 cm}}
\footnotesize
ورودی ۱۴۰۰ مهندسی کامپیوتر دانشگاه اصفهان
\end{tabular}
\end{document}